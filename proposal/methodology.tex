\chapter{Methodology}
As first objective, the dataset will be created using the feature extractor developed by 
\cite{rasthofer2014machine}. The extractor creates meaningful information using the Android methods names and 
their real implementation in the Android API. As result, the extractor gives the method class, if has been hand 
annotaded, and a list of features. These features are semantic and syntactic features, containing information about 
the method name, parameters, return type, method and classes modifiers, class modifiers, if exists data flow in the 
method return or parameters and the required permissions.

The extractor will be used in many Android APIs as possible, since there are a low quantity of hand annotated 
methods, it is important to extract as many methods from classes as possible. It is possible to have duplicated 
methods at the end of this evaluation due to backward compatibility, so, methods from older APIs will be 
overwritten.

The methods used in the comparison will be SVM, Naive Bayes, Decision Trees, MLP, KNN, and ensemble classifications 
methods, which will be evaluated in 30 datasets sampled from the original. Each of these datasets are 
subdivided into train dataset and test dataset, containing 80\% of the original dataset for model training and 20\% 
for test the model effectiveness. They must maintain the classes proportion observed in the original dataset, if 
the original has 45\% of source methods, the train and test must have a proportion close to that. Create these 
datasets are important to make a Hypothesis Test, this help to statically evaluate if a classifier is really 
effective when applied in this dataset. The proposed evaluation flow is shown in figure \ref{flow}.

\begin{multicols}{2}%
    \noindent%
    {%
    \begin{equation} \label{precision} precision = \frac{TP}{TP+FP} \end{equation}%
    \begin{equation} \label{accuracy} accuracy = \frac{TP+TN}{TP+TN+FP+FN} \end{equation}%
    }%
    {%
    \begin{equation} \label{recall} recall = \frac{TP}{TP+FN} \end{equation}%
    \begin{equation} \label{f1} F1 = \frac{2 \times recall \times precision}{recall + precision} \end{equation}%
    }%
\end{multicols}

Each classifier will be evaluated using precision, accuracy, recall and F1 score. The precision, equation 
\ref{precision}, is the ratio of correctly predictions to the total predictions done. Accuracy, equation 
\ref{accuracy}, is the ratio of correctly true predictions to the total of true predictions. Recall, equation 
\ref{recall} is the ratio of correctly positive predictions to all the predictions of a class. F1, equation 
\ref{f1}, score represents the harmonic mean between precision and recall \cite{sasaki2007truth}.

We can rewrite precision, accuracy, recall and F1 score in the equations above, using true positives (TP), true 
negatives (TN), false positives (FP) and false negatives (FN). True positives are all instances of a class $C$ that 
are correctly classified. True negatives are all instances that do not belong to $C$ and are correctly classified. 
False positives of a class $C$ are the instances of other classes that has been classified as $C$. False negatives 
are instances of $C$ that has been classified as not belonging to $C$.

\begin{figure}[!h]
    \begin{center}
        \begin{tikzpicture} [node distance=.9cm, start chain=going below]
    \node[punktchain] (ext) {Feature extractor};
    \node[punktchain, join] (dset) {Dataset};
    \node[punktchain, below left = of dset] (sdset2) at (0.5, -3) {Sampled Dataset 2};

    %We need to redefine the join-style to have the -> turn out right
    \begin{scope}[start branch=venstre, every join/.style={->, thick, shorten <=1pt}]
        \node[punktchain, on chain=going left] (sdset1) {Sampled Dataset 1};
    \end{scope}

    \node[punktchain, right = of sdset2] (dots) {$\cdots$};

    \begin{scope}[start branch=hoejre]
        \node[punktchain, on chain=going right] (sdset30) {Sampled Dataset 30};
    \end{scope}
    
    \node[punktchain, below = of dset] (class) at (0, -5.5) {Classifiers};
    % Now that we have finished the main figure let us add some "after-drawings"
    %% First, let us connect (finans) with (disk). We want it to have
    %% square corners.
    \draw[|-,-|,->, thick,] (sdset1.south) |-+(0,-1em)-| (class.north);
    \draw[|-,-|,->, thick,] (sdset30.south) |-+(0,-1em)-| (class.north);
    \draw[|-,-|,->, thick,] (dots.south) |-+(0,-1em)-| (class.north);
    \draw[|-,-|,->, thick,] (sdset2.south) |-+(0,-1em)-| (class.north);
    % Now, let us add some braches. 
    %% No. 1
    \draw[tuborg] let \p1=(sdset1.west), \p2=(sdset30.east) in ($(\x1,\y1+2.5em)$) -- ($(\x2,\y2+2.5em)$);

    \begin{scope}
        \node[punktchain, above left = of ext] (meth) at (0, 1) {Android Methods};
        \node[punktchain, above right = of ext] (api) at (0, 1) {Android APIs};

        \draw[|-,-|,->, thick,] (meth.south) |-+(0,-1em)-| (ext.north);
        \draw[|-,-|,->, thick,] (api.south) |-+(0,-1em)-| (ext.north);
    \end{scope}

    \node[punktchain, below = of class] (res) {Results};
    \draw[|-,-|,->, thick,] (class.south) |-+(0,-1em)-| (res.north);
\end{tikzpicture}
    \end{center}
    \caption{%
        Overview of the proposed evaluation flow. First, the feature extractor uses Android Methods and %
        Android APIs to create a dataset, which will be sampled into other 30 different datasets that will be used %
        to evaluate the Classifiers.%
    }\label{flow}
\end{figure}

