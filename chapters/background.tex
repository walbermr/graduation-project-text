\chapter{Background}\label{background}

In this section, we will introduce the techniques used in flow analysis and the issues of having a poor list of source and sink methods to make the analysis.

Dynamic Flow Enforcement are techniques that tracks and enforces information flow during the application runtime. These methods relies on Taint Analysis to track possible sensitive data flow to untrusted sinks. Taint Analysis marks every sensitive data gathered from a source and every other variable that inherit any operation from the tainted data, in the end, if any tainted variable is accessed by a sink method, the information has leaked and the analysis gives a detailed path through which the data passed. During the tracking, there are different methods to enforce in runtime that the data will not leak, \cite{fernandes2016flowfence} uses virtualization to guarantee that the data will only operate in the controlled environment and \cite{sun2017data} declassifying information before it is computed in trusted methods or if reach a trusted API.

Static Flow Enforcement starts by creating abstract models of the application code to provide a simpler representation \citep{myers1999jflow}, using frameworks like Soot \citep{vallee2000optimizing}. Then, this model will be used in control-flow, data-flow and points-to analysis to observe the application control, data sequence and compute static abstractions for variables \cite{li2017static}. These methods are implemented and used in DroidSafe \cite{gordon2015information}. JFlow \cite{myers1999jflow} inserts statically checked and secured code when the application computes on sensitive data. There are also Static Enforcement techniques that uses taint analysis, like \cite{arzt2014flowdroid} that created a precise method of static information flow tracking using taint analysis.

Both Static and Dynamic Flow Enforcement techniques require information of which methods is a source of sensitive data and which is a data sink. This is used to identify if a sink method is truly leaking sensitive data or not. So,lists containing sources and sinks of sensitive data are hand created, but this solution is impractical considering a huge API like the Android API \cite{rasthofer2014machine}.

Considering that issue, \cite{rasthofer2014machine} proposed to use machine learning to automatically create a categorized list of sources and sinks methods to be used in Flow Enforcement techniques. The list consists in methods classified into Flow Classes and Android Method Categories. The Flow Classes are source of sensitive data, or just source, and sink of data, but also, the method can be neither source or sink. For Android Methods Categories, there are 12 different classes: account, Bluetooth, browser, calendar, contact, database, file, network, NFC, settings, sync, a unique identifier, and no category if the method does not belong to any of the previous.

The authors shortly compared Decision Trees and Naive Bayes with the SVM and choosed to use SVM to create the categorized list of sources and sinks, as SVM showed to be more precise in categorize the Android methods. 

To classify, the authors utilize features extracted from the methods, like the method name, if the method has parameters, the return value type, parameter type, if the parameter is an interface, method modifiers, class modifiers, class name, if the method returns a value from another source method, if one parameter flows into a sink method, if a method parameters flows into a abstract sink and the method required permission. 

To categorize the methods, were used features like class name, method invocation, body contents, parameter type and return value type. After that, the methods list is generated containing if it is a sink, source and the method category.

