\section{Monolithic}\label{mono}

For the Monolithic classifiers, we have the Decision Tree, Multinomial and Bernoulli Naive Bayes, K-Nearest Neighbor (KNN), Support Vector Machine (SVM) , Multi-Layer Perceptron  and Random Forest.

The setup for Decision Tree is the split strategy to select the most relevant feature with Gini Impurity as criterion. In Multinomial and Bernoulli Naive Bayes we choose a $\alpha$ of 1. For KNN, we use 5 as the number of neighbors and a uniform weight for the neighbor points. For MLP we used L2 regularization with alpha of $10^{-4}$, 1000 neurons in hidden layer, no early stopping, activation layer as relu, max of 1000 iterations and tolerance of $10^{-4}$. For Random Forests we are using 10 estimators, the minimum samples to split is 2 and Gini Impurity as criterion.

\begin{table}[h!]
    \centering
    \renewcommand{\arraystretch}{1.8}
    \begin{tabular}{ |p{3cm}|p{2.6cm}p{2.6cm}p{2.6cm}p{2.6cm}| }
        \hline
        Model & Precision & Recall & F1 Score & Accuracy \\
        \hline
        Decision Tree & 0.8474(0.0505) & 0.8388(0.0549) & 0.8410(0.0357) & 0.8497(0.0217) \\
        Multinomial NB & 0.8297(0.0595) & 0.8211(0.0552) & 0.8231(0.0416) & 0.8387(0.0282) \\
        Bernoulli NB & 0.8307(0.0609) & 0.8180(0.0561) & 0.8220(0.0424) & 0.8378(0.0271) \\
        KNN & 0.8635(0.0547) & 0.7991(0.0657) & 0.8217(0.0484) & 0.8432(0.0307) \\
        Linear SVM & 0.8920(0.0559) & 0.8718(0.0539) & 0.8796(0.0436) & 0.8859(0.0284) \\
        MLP & 0.8936(0.0547) & \textbf{0.8759(0.0541)} & \textbf{0.8830(0.0437)} & \textbf{0.8908(0.0302)} \\
        Random Forest & \textbf{0.8953(0.0568)} & 0.8522(0.0605) & 0.8684(0.0438) & 0.8791(0.0309) \\ [1ex]
        \hline
        \end{tabular}
        \caption{%
        Result for Monolithic Classifiers, Mean(Standard Deviation), the best classifier for each metric is highlighted in bold.%
        }\label{mcs_perceptron_table}
\end{table}

We can observe that the MLP has better results overall, just behind of Random Forest in precision by 0,0017. This interval is within the confidence interval, the standard deviation of MLP and Random Forest have the same approximation of 0.05. We can also observe that the SMV is close to MLP and Random Forest in all metrics, this is an evidence that SVM can solve this problem as well as these two.