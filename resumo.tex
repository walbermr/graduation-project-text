Um programa computa em dados sensíveis e não sensíveis, esses dados seguem um fluxo específico indo de \textit{data sources} para \textit{data sinks}. O vazamento de dados acontece quando dados sensíveis chegam sem autorização em \textit{sinks}, para prevenir isso,  técnicas estáticas e dinâmicas de \textit{Flow Enforcement} garantem que esses dados não cheguem nessas \textit{sinks}. Para isso, esses métodos usam listas, geradas manualmente, de métodos que sejam \textit{sources} sensíveis ou \textit{sinks}, e essa solução é impraticável para grandes APIs como a do Android. Visto isso, uma abordagem usando \textit{machine learning} foi desenvolvida para classificar esses métodos \textit{sources} e \textit{sinks}. \\

O presente trabalho tem como objetivo criar um dataset para avaliar os métodos de classificação Monolítios e Sistemas Multiclassificadores para decidir qual é o mais apropriado para esse problema. Para criar o dataset, foram utilizados as APIs Android Level 3 a Level 27, excluindo as de Level 18 e 20, abrangendo a maioria das APIs recentes que foram aplicadas em um extrator de features. Para realizar a classificação, foram amostrados aleatoriamente 30 datasets de treino e teste, foram avaliados precisão, recall, F1 score e acurácia dos classificadores em cada um dos datasets e feito um teste de hipótese. Neste trabalho concluimos que o MLP tem melhores resultados no problema de classificação de métodos Android, METADES com árvore de decisão e SVM também tem um resultado satisfatório e ficam estatisticamente empatados com o MLP.
